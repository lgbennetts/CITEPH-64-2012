\documentclass[12pt,a4paper]{article}
\oddsidemargin=-5.00mm
\textwidth=170.00mm
\topmargin=-10.00mm
\textheight=240.00mm
\usepackage{amsmath}
\usepackage{amssymb}
\usepackage{mathrsfs}
%\usepackage{eufrak}
\usepackage{amssymb,graphicx,psfrag}
\usepackage{color}  %%%%% for colour
\usepackage{rotating}
%\usepackage{natbib}
% \usepackage{refcheck}
% \usepackage{overpic}

\newcommand{\real}{\textrm{Re}}
\newcommand{\imag}{\textrm{Im}}
\newcommand{\wrt}{ ~ {\rm d}}
\newcommand{\pd}[2]{\frac{\partial #1}{\partial #2}}
\newcommand{\sech}{\rm sech}
\newcommand{\cosech}{\rm cosech}
\newcommand{\bnab}{{\bf \nabla}}
\newcommand{\besj}{\textrm{J}}
\newcommand{\besy}{\textrm{Y}}
\newcommand{\besh}{\textrm{H}}
\newcommand{\sgn}{\textrm{sgn}}
\newcommand{\mn}{\textrm{mn}}
\newcommand{\mx}{\textrm{mx}}
\newcommand{\diag}{\textrm{diag}}
\renewcommand{\vec}{\textrm{vec}}
\newcommand{\mat}{\textrm{Mat}}
\newcommand{\inv}{\textrm{inv}}
\def\ci{{\mathrm i}}
\renewcommand{\exp}{{\rm e}}
\newcommand{\pn}{\partial_{n}}
\newcommand{\ps}{\partial_{s}}
\newcommand{\px}{\partial_{x}}
\newcommand{\pz}{\partial_{z}}
\newcommand{\order}{\textrm{O}}
\newcommand{\lap}{\nabla^{2}}
\newcommand{\eps}{\varepsilon}

% - Gen Symbols & fonts

\newcommand{\ca}{\mathcal}
\newcommand{\bs}{\boldsymbol}
\newcommand{\cc}{\overline}
\newcommand{\ds}{\displaystyle}

\newcommand{\newbit}[1]{{\color{blue}\bf #1}}
\newcommand{\qu}[1]{{\color{red}\it #1}}

\newcommand{\eg}{e.g.\ }
\newcommand{\ie}{i.e.\ }
\newcommand{\etal}{\textit{et al.}\ }

% - particular to this problem (vary easily)

\newcommand{\dom}{\Omega }
\newcommand{\bv}{{\bf v}}
\newcommand{\vp}{{\phi}}
\newcommand{\vph}{{\varphi}}
\newcommand{\vpr}{{\varphi}}
\newcommand{\fp}{{\sigma}}
\newcommand{\vm}{{\zeta}}
\newcommand{\ang}{{\psi}}
\newcommand{\ym}{{\eta}}
\newcommand{\nm}{{\bf n}}
\newcommand{\bdy}{\Gamma}
\newcommand{\rad}{a}
\newcommand{\kx}{\alpha}
\newcommand{\ky}{\beta}
\newcommand{\xo}{\check{x}}
\newcommand{\yo}{\check{y}}
\newcommand{\bx}{{\bf x}}
\newcommand{\bxo}{\check{\textbf{x}}}
\newcommand{\bu}{\textbf{u}}
\newcommand{\bfI}{\textbf{f}}
\newcommand{\so}{\check{s}}
\newcommand{\tho}{\check{\theta}}
\newcommand{\force}{f}
\newcommand{\cent}{\textbf{c}}

\begin{document}
\title{A three-dimensional model of wave scattering by floating disks in a wave tank.}
\author{L. Bennetts}
\date{\today}
\maketitle

\section{Preliminaries}

Consider a rectangular wave tank, with geometry defined by a Cartesian co-ordinate system $x$, $y$, $z$, where the co-ordinates are parallel to the length, breadth and depth of the tank, respectively.
The depth co-ordinate $z$ points upwards and has its origin set to coincide with the surface of the fluid in the tank when at rest. 
The fluid in the tank occupies the domain
\begin{equation}
(x,y,z)\in 
\dom\times[-h,0]
\quad
\text{where}
\quad
\Omega
=
\{
x,y:
0<x<l,
0<y<w,
\}.
\end{equation}
The equilibrium position of the wave maker is the surface $x=0$ and the absorbing beach is $x=l$.
The tank floor ($z=-h$) and the side walls ($y=0,w$) are all rigid and impenetrable. 

Under the usual assumptions of linear motions, the fluid velocity field $\bv(x,y,z,t)$ is described by a complex-valued velocity potential $\vp(x,y,z)$ via $\bv=\real\{(g/\ci\omega)(\nabla,\partial_{z})\vp\exp^{-\ci\omega t}\}$.
Here $\nabla\equiv(\partial_{x},\partial_{y})$ is the horizontal differential operator, $g\approx 9.81$\,m\,s$^{-2}$ is acceleration due to gravity, and $\omega$ is a prescribed angular momentum.
The velocity potential satisfies Laplace's equation throughout the  fluid domain, \ie
\begin{equation}
(\lap+\partial^2_{zz})
\vp
=
0
\qquad
(x,y,z)\in\dom\times[-h,0].
\end{equation}
The fluid velocity normal to the impermeable walls of the tank vanish, yielding the conditions
\begin{equation}
\partial_{y}\vp = 0
\quad
(y=0,w),
\quad
\text{and}
\quad
\partial_{z}\vp = 0
\quad
(z=-h).
\end{equation}
The free surface condition is
\begin{equation}
\partial_{z}\vp
=
\fp\vp
\qquad
(z=0),
\end{equation}
where $\fp=\omega^{2}/g$ is a frequency parameter.
The boundary conditions at the wave maker and the beach are as yet unknown, and will be disregarded for the time being. 
 
Separation solutions are sought in which the vertical motion is assumed to be independent of the horizontal motions.
The full solution is then a superposition of such solutions, and is expressed as
\begin{equation} 
\vp(x,y,z)
=
\sum_{n=0}^{\infty}
\vph(\bx :k_{n})
\vm(z:k_{n})
,
\;\;
\text{where}
\;\;
\vm(z:k_{n})=\cosh\{k_{n}(z+h)\}
\;\;
\text{and}
\;\;
\bx
=
(x,y).
\end{equation}
The quantities $k_{n}$ are roots $k$ of the dispersion relation
\begin{equation}
k\tanh(kh)=\fp.
\end{equation}
The primary root $k_{0}=2\pi/\lambda\in\mathbb{R}$ is the free-surface wave number, and $\lambda$ is the corresponding wavelength.
The subsequent roots $k_{n}\in\ci\mathbb{R}$ $(n=1,2,\dots)$  are ordered in ascending magnitude.

The horizontal functions $\vph$ satisfy Helmholtz equation
\begin{equation}
\nabla^{2} 
\vph
+
k_{n}^{2}\vph
=0,
\end{equation}
with the boundary conditions $\partial_{y}\vph=0$ for $y=0,w$.
Separable solutions of the form $\vph=X(x)Y(y)$ exist, with
\begin{equation}
X(x)
=
a\exp^{\ci\kx_{m}(k_{n}) x}
+
b\exp^{-\ci\kx_{m}(k_{n}) x}
,
\quad
Y(y)
=
\cos(\ky y)
,
\end{equation}
for $m=0,1,2,\dots$, where $\ky_{m}=m\pi/w$ and $\kx_{m}(k)=\sqrt{k^{2}-\ky_{m}^{2}}$.
The amplitudes $a$ and $b$ are undefined.
If $k_{n}\in\ci\mathbb{R}$, \ie $n\geq 1$, then $\kx_{m}\in\ci\mathbb{R}$, which implies that the corresponding motions they support are, as expected, evanescent. 
For $k_{0}$ two types of motions exist.
If $k_{0}\geq \ky_{m}$, for some $m$, then $\kx_{m}\in\mathbb{R}$. The supported motion is thus a sum of two propagating waves, travelling in directions $\pm\ang=\pm\arccos(\kx_{0,m}/k_{0})$ with respect to the positive $x$ axis.
Let $P-1$ be the largest natural number for which $k_{0}\geq \ky_{P}=P\pi/w$. Then the motion will be said to have $P$ propagating wave components.
For $m\geq P$ the wave number $k_{0}<\ky_{m}$ and $\ky_{m}\in\ci\mathbb{R}$, so that the motion is, again, evanescent.

The full solution is now expressed as
\begin{equation}
\label{eq:PotOW}
\begin{array}{rcl}
\vp(x,y,z)
&
=
&
\ds
\sum_{n=0}^{\infty}
\sum_{m=0}^{\infty}
\Big(
a_{n,m}\exp^{\ci\kx_{m}(k_{n}) x}
+
b_{n,m}\exp^{-\ci\kx_{m}(k_{n}) x}
\Big)
\ym_{m}(y)
\vm(z:k_{n}),
\end{array}
\end{equation}
where $\ym_{m}=\cos(\ky_{m}y)$.
It will be useful to employ the decomposition $\vp=\vp_{p}+\vp_{ev}$, where
\begin{equation}
\begin{array}{rcl}
\vp_{p}(x,y,z)
&
=
&
\ds
\sum_{m=0}^{P-1}
\Big(
a_{0,m}\exp^{\ci\kx_{m}(k_{0}) x}
+
b_{0,m}\exp^{-\ci\kx_{m}(k_{0}) x}
\Big)
\ym_{m}(y)
\vm(z:k_{0}),
\end{array}
\end{equation}
and
\begin{equation}
\begin{array}{rcl}
\vp_{e}(x,y,z)
&
=
&
\ds
\sum_{m=P}^{\infty}
\Big(
a_{0,m}\exp^{\ci\kx_{m}(k_{0}) x}
+
b_{0,m}\exp^{-\ci\kx_{m}(k_{0}) x}
\Big)
\ym_{m}(y)
\vm(z:k_{0})
\\[8pt]
&
&
\ds
+
\sum_{n=1}^{\infty}
\sum_{m=0}^{\infty}
\Big(
a_{n,m}\exp^{\ci\kx_{m}(k_{n}) x}
+
b_{n,m}\exp^{-\ci\kx_{m}(k_{n}) x}
\Big)
\ym_{m}(y)
\vm(z:k_{n}),
\end{array}
\end{equation}
are the propagating and evanescent parts, respectively.

For the proposed experiments, the width of the tank is 16\,m, and the wavelengths with be in the range 0.5\,m to 6.25\,m.
The number of propagating wave components will therefore range from six to 65.
For a transmitted wave field, consisting of $P$ propagating wave components, in theory $2P$ independent measurements must be made to calculate the amplitudes.
This assumes no reflected waves.

%%% OPEN WATER %%%

\section{Solution in the open water}
\label{sec2}

%%% Green's function %%%

\subsection{Green's functions}

Consider the  Green's function $ G(\bx\vert \bxo: k)$, where $\bxo=(\xo,\yo)$,
which satisfies the governing equations
\begin{equation}
\lap
G
+
k^{2}
G
=
\delta(x-\xo)
\delta(y-\yo),
\qquad
x\in \mathbb{R},
y\in(0,w)
,
\end{equation}
the boundary conditions $\partial_{y}G_{n}=0$ on $y=0,w$.
A radiation condition, which states that the Green's function represents outgoing waves as $x\to\pm\infty$, is also imposed. 

The Green's functions can be obtained by analogy to a periodic Green's function, $\hat{G}(x,y\vert \xo,\yo: k)$, which satisfies the governing equation
\begin{equation}
\lap
\hat{G}
+
k^{2}
\hat{G}
=
\delta(x-\xo)
\delta(y-\yo)
+
\delta(x-\xo)
\delta(y+\yo)
\qquad
x\in \mathbb{R},
y\in(-w,w)
,
\end{equation}
periodic boundary conditions
\begin{equation}
[\hat{G}]_{y=-w}=[\hat{G}]_{y=w}
,
\qquad
[\hat{G}_{y}]_{y=-w}=[\hat{G}_{y}]_{y=w}
,
\end{equation}
and radiation conditions.
The solution to this problem is a
superposition of regular quasi-periodic Green's functions (e.g. Bennetts \& Squire \cite{Ben&Squ09a}) at the two source points, with the phase-change parameter set to zero.
Hence
\begin{subequations}\label{eqns:Greens}
\begin{equation}
\begin{array}{rcl}
\hat{G}(x,y\vert \xo,\yo:k)
&=&
\ds
\frac{1}{4\ci w}
\left(
\sum_{m=-\infty}^{\infty}
\frac{1}{\kx_{m}(k)}
\exp^{\ci \kx_{m}(k) \vert x-\xo\vert - \ci \beta_{m}(y-\yo)}
+
\sum_{m=-\infty}^{\infty}
\frac{1}{\kx_{m}(k)}
\exp^{\ci \kx_{m}(k) \vert x-\xo\vert - \ci \beta_{m}(y+\yo)}
\right)
\\[12pt]
&=&
\ds
\frac{1}{2\ci w}
\sum_{m=-\infty}^{\infty}
\frac{1}{\kx_{m}(k)}
\exp^{\ci \kx_{m}(k) \vert x-\xo\vert-\ci \beta_{m} \yo}
\eta_{m}(y)
\\[12pt]
&=&
\ds
\frac{1}{2\ci w}
\sum_{m=0}^{\infty}
\frac{1}{\kx_{m}(k)}(2-\delta_{m0})
\exp^{\ci \kx_{m}(k) \vert x-\xo\vert}
\eta_{m}(\yo)\eta_{m}(y),
\end{array}
\end{equation}
noting that $\kx_{m}=\kx_{-m}$.
The required Green's function, $G$, is therefore
\begin{equation}
G(\bx\vert \bxo,\yo:k)
=
\hat{G}(\bx\vert \bxo,\yo:k).
\end{equation}
\end{subequations}

%%% INTEGRAL EQNS %%%

\subsection{Integral equations}

Applying Green's theorem in the plane to the potential $\vph$ and the Green's function $G$ produces the integral expression
\begin{equation}\label{eqn:intexp-i}
\eps(\bx)\vph(\bx)
=
\ds
\force(\bx)
-
\sum_{d=1}^{D}
%\rad_{d}
\nm_{d}\cdot
\int_{\bdy_{d}}\wrt s_{d} \,
\Big\{
\vph_{n}(\bxo_{d})
\bnab
G(\bxo_{d}\vert \bx)
-
G(\bxo_{d}\vert \bx)
\bnab
\vph(\xo,\yo)
\Big\}
%_{(\xo,\yo)\in\bdy_{d}}
.
\end{equation}

Here $D$ denotes the number of disks in the tank, $\bdy_{d}$ denotes the  boundary of disk $d$, and $\nm_{d}$ and $s_{d}$ denote the (outward) normal to $\bdy_{d}$ and its arclength, respectively.
Coordinate subscripts are used to  indicate evaluation on the boundary of a particular disk, \ie $\bx_{d}\equiv[\bx ]_{\bdy_{d}(s_{d})}$.
The quantity $\eps(\bx)$ is equal to unity if the field point $\bx$ is in the open-water region, one-half if the field point is on a disk edge, and zero if the field point is in the disk-covered region.
The function $\force$ is the wave forcing, defined by 
$\force(\bx:k)=I\exp^{\ci kx}$, where $I=I(k)$ is a prescribed amplitude, which will generally be assumed to be $I(k_{n})=\delta_{0,n}$.

For the circular disks considered in the experiments,  the radius of disk $d$ is defined to be $\rad_{d}$, and its local polar coordinate system is $(r_{d},\theta_{d})$, with an origin located at the centre of the disk, denoted %$(x_{d},y_{d})$ 
$\cent_{d}\equiv (x_{d}^{(c)},y_{d}^{(c)})$, which is related to the horizontal Cartesian coordinates via $x-x_{d}^{(c)}=r_{d}\cos\theta_{d}$ and $y-y_{d}^{(c)}=r_{d}\sin\theta_{d}$.
In this case, the integral expression (\ref{eqn:intexp-i}) becomes
\begin{equation}\label{eqn:intexp_ii}
\eps(\bx)\vph(\bx)
=
\force(\bx)
-
\sum_{d=1}^{D}
\rad_{d}
\int_{-\pi}^{\pi}\wrt \tho \,
\Big\{
\vph(\bxo)
\partial_{r_{d}}
G(\bxo_{d}\vert \bx)
-
G(\bxo_{d}\vert \bx)
\partial_{r_{d}}
\vph(\bxo_{d})
\Big\}
%_{(\xo,\yo)=(x_{d},y_{d})+\rad_{d}(\cos\tho,\sin\tho)}
.
\end{equation}

A system of integral equation are obtained from (\ref{eqn:intexp_ii}) by allowing the field point to tend to each of the $D$ boundaries $\bdy_{d}$ in succession, to give
\begin{equation}\label{eqn:IEsys}
\frac12
\vpr_{j}(\theta)
=
\force_{j}(\theta)
-
\sum_{d=1}^{D}
\rad_{d}
\int_{-\pi}^{\pi}
\wrt \tho\,
\Big\{
\vpr_{d}(\tho)
\partial_{r_{d}}
G(\bxo_{d}\vert \bx_{j})
-
G(\bxo_{d}\vert \bx_{j})
\vpr'_{d}(\tho)
\Big\}
%_{\substack{
%(\xo,\yo)=(x_{d},y_{d})+\rad_{d}(\cos\tho,\sin\tho) \\
%(x,y)=(x_{j},y_{j})+\rad_{j}(\cos\theta,\sin\theta)} }
,
\end{equation}
for $j=1,\dots,D$.
In the above the following short-hand notations have been used:
%\begin{equation}
%\vpr_{d}(\theta) \equiv \left[\vph(x,y)\right]_{(x,y)=(x_{d},y_{d})+\rad_{d}(\cos\theta,\sin\theta)},
%\end{equation}
%\begin{equation}
%\vpr'_{d}(\theta) \equiv \left[\partial_{r_{d}}
%\vph(x,y)\right]_{(x,y)=(x_{d},y_{d})+\rad_{d}(\cos\theta,\sin\theta)},
%\end{equation}
%and
%\begin{equation}
%\force_{d}(\theta) \equiv \left[\vph_{i}(x,y)\right]_{(x,y)=(x_{d},y_{d})+\rad_{d}(\cos\theta,\sin\theta)}.
%\end{equation}
\begin{equation}
\vpr_{d}(\theta) \equiv \vph(\bx_{d}(\theta)),
\quad
\vpr'_{d}(\theta) \equiv \partial_{r_{d}}(\bx_{d}(\theta)),
\quad
\text{and}
\quad
\force_{d}(\theta) \equiv \vph_{i}(\bx_{d}(\theta)).
\end{equation}

%%% NUMERICS

\subsection{Numerical evaluation of integral equations}

%The subscripts that identify a quantity to a particular disk will be dropped in this section.

\subsubsection{Galerkin technique}

The Galerkin technique is applied to solve the system of integral equations (\ref{eqn:IEsys}).
Specifically, the unknowns $\vpr_{d}$ and $\vpr'_{d}$ are expanded in the Fourier series
\begin{equation}
\vpr_{d}(\theta)
=
\sum_{q=-Q}^{Q}
\vpr_{d,q}
\exp^{\ci q\theta}
\quad
\text{and}
\quad
\vpr'_{d}(\theta)
=
\sum_{q=-Q}^{Q}
\vpr'_{d,q}
\exp^{\ci q\theta}
\quad
(d=1,\dots,D),
\end{equation}
where the value of $Q$ is chosen to be large enough to achieve the required accuracy.
Inner products of each of the integral equations are then taken with respect to each of the test functions 
$\exp^{-\ci q \theta}/2\pi$ $(q=-Q,\dots,Q)$ in turn to generate a system of linear equations.
%The system will be written using array notation.
For a particular disk $j$, the system generated by the Galerkin technique is expressed in array notation as
\begin{equation}\label{eqn:GalSys}
\frac{1}{2}
\bu_{j}
=
\bfI_{j}
-
\sum_{d=1}^{D}
\left(
G'_{j,d}
\bu_{d}
-
G_{j,d}
\bu'_{d}
\right)
.
\end{equation}
In this expression, the vectors $\bu_{d}$, $\bu'_{d}$ and $\bfI_{d}$ are defined as
\begin{equation}
\bu_{d} =  
\vec\{
\vpr_{d,q}
:
q=-Q,\dots,Q\}
%\left(
%\begin{array}{c}
%\vpr_{d,-Q}
%\\
%\vdots
%\\
%\vpr_{d,Q}
%\end{array}
%\right)
,
\quad
\bu_{d}' =  
\vec\{
\vpr_{d,q}'
:
q=-Q,\dots,Q\},
\end{equation}
%\left(
%\begin{array}{c}
%\vpr_{d,-Q}'
%\\
%\vdots
%\\
%\vpr_{d,Q}'
%\end{array}
%\right)
%,
%\quad
and
\begin{equation}
\bfI_{d}
=
\vec
\left\{
\langle
\force_{j}
\rangle_{q}
:
q=-Q,\dots,Q
\right\}
%\frac{1}{2\pi}
%\left(
%\begin{array}{c}
%\int_{-\pi}^{\pi}\force_{j}\exp^{\ci Q \theta}\wrt\theta
%\\
%\vdots
%\\
%\int_{-\pi}^{\pi}\force_{j}\exp^{-\ci Q \theta}\wrt\theta
%\end{array}
%\right)
=
,
\end{equation}
using the inner-product notation
\begin{equation}
\langle 
f
\rangle_{q}
\equiv
\frac{1}{2\pi}
\int_{-\pi}^{\pi}
f(\theta)\exp^{-\ci q \theta}\wrt\theta.
\end{equation}
The matrices $G_{j,d}$ and $G'_{j,d}$ are defined by
\begin{equation}
G_{j,d}
=
\mat
\left\{
\langle\langle
G(\bxo(\theta)\vert\bx(\tho))
\rangle\rangle_{q,p}
:
\text{rows/columns }
p,q=-Q,\dots,Q
\right\}
\end{equation}
and
\begin{equation}
G_{j,d}'
=
\mat
\left\{
\langle\langle
\partial_{r_{d}}
G(\bxo(\theta)\vert\bx(\tho))
\rangle\rangle
:
\text{rows/columns }
p,q=-Q,\dots,Q
\right\},
\end{equation}
where
\begin{equation}
\langle\langle
G
\rangle\rangle
\equiv
\frac{1}{2\pi}
\int_{-\pi}^{\pi}
\int_{-\pi}^{\pi}
G(\tho\vert \theta)
\exp^{\ci(q\tho -p\theta)}
\wrt\tho\wrt\theta.
\end{equation}


%

\subsubsection{A relationship between $G$ and $\partial_{r}G$}

It can be shown that the Green's function for an unbounded domain, \ie
\begin{equation}
G_{0}(\bx,\bxo:k)
=
\frac{1}{4\ci}
\besh_{0}(k\vert\bx-\bxo\vert)
,
\end{equation}
satisfies the  identity
%\begin{equation}\label{eqn:Grn0Id}
%\int_{-\pi}^{\pi}
%\wrt \tho
%\left[
%\left\{
%\besj_{n}(k\rad)
%\left(
%\partial_{r}
%G_{0}(\bx,\bxo:k)
%\right)
%-
%k\besj_{n}'(k\rad)
%G_{0}(\bx,\bxo:k)
%\right\}
%\exp^{\ci n\tho}
%\right]_{\bxo=\rad(\cos\theta,\sin\theta)}
%=
%\left\{
%\begin{array}{c c}
%\frac{1}{2\rad}\besj_{n}(k\rad)
%&
%(r=\rad),
%\\
%0
%&
%(r>\rad).
%\end{array}
%\right.
%\end{equation} 
\begin{equation}\label{eqn:Grn0Id}
\besj_{n}(k\rad)
\langle
\partial_{r}
G_{0}
\rangle_{q}F
-
k\besj_{q}'(k\rad)
\langle
G_{0}
\rangle_{q}
=
\left\{
\begin{array}{c c}
\frac{1}{4\pi\rad}\besj_{q}(k\rad)
&
(r=\rad),
\\
0
&
(r>\rad).
\end{array}
\right.
\end{equation} 
Using the expression for the quasi-periodic Green's function, $\hat{G}$, as a sum of images, \ie
\begin{equation}
\hat{G}(\bx,\bxo)
=
\sum_{j=-\infty}^{\infty}
G_{0}(\bx,\bxo+(0,2jw))
,
\end{equation}
it can be shown that $G(\bx\vert\bxo:k)$ also satisfies identity (\ref{eqn:Grn0Id}).
It follows that
\begin{equation}
G_{j,d}'
J_{d}
-
G_{j,d}
J'_{d}
=
\left\{
\begin{array}{c c}
\frac{1}{2}J_{d}
&
(d=j),
\\
0
&
(d\neq q).
\end{array}
\right.
\end{equation}
where $J_{d}=\diag\{\besj_{-Q}(k\rad_{d}),\dots,\besj_{Q}(k\rad_{d})\}$ and 
$J'_{d}=\diag\{k\besj'_{-Q}(k\rad_{d}),\dots,k\besj'_{Q}(k\rad_{d})\}$.
The system of equation (\ref{eqn:GalSys}) therefore simplifies to
\begin{equation}\label{eqn:GalSys_ii}
\bu_{j}
=
\bfI_{j}
-
\sum_{d=1}^{D}
G_{j,d}
\left(
\hat{J}_{d}
\bu_{d}
-
\bu'_{d}
\right)
,
\quad
\hat{J}_{d}
=
J'_{d}J_{d}^{-1}
.
\end{equation}
The advantage of the new expression is that only integrals involving the Green's function itself, and not it's derivative need to be calculated numerically.
The method was originally used by Bennetts \& Squire \cite{Ben&Squ09b} for a related problem. 

%

\subsubsection{Kummer method}

The series expression for the Green's function given in equations (\ref{eqns:Greens}) is not suitable for numerical calculations due to slow convergence around the logarithmic singularity when the field as source points tend to one another.
To improve the convergence a Kummer transformation is applied, in which a second series, with the same asymptotic behaviour as $m\to\infty$, is subtracted from the Green's function and added back on in closed form.
Consider then, the Greens function as the sum of two series 
\begin{equation}
G(\bx\vert\bxo)
=
\frac{1}{4\ci w}
\sum_{\pm}
\sum_{m=\infty}^{\infty}S_{m\pm},
\quad
\text{where}
\quad
S_{m\pm}
=
\frac{1}{\alpha_{m}}
\exp^{\ci\alpha_{m}\vert x-\xo\vert - \ci\beta_{m}(y\pm\yo)}
.
\end{equation}
As $m\to\pm\infty$ $\alpha_{m}\to\ci\pm\beta_{m}$, and
the appropriate series to subtract are therefore
\begin{equation}
\frac{1}{4\ci w}
\sum_{m=1}^{\infty}
\left(
T_{m\pm}
+
T_{-m\pm}
\right)
,
\quad
\text{where}
\quad
T_{m\pm}
=
\frac{1}{\ci\beta_{\vert m \vert}}
\exp^{-\beta_{\vert m\vert}\vert x-\xo\vert - \ci\beta_{m}(y\pm\yo)}.
\end{equation}
The closed form for these series are
\begin{equation}\nonumber
\frac{1}{4\ci w}
\sum_{m=1}^{\infty}
\left(
T_{m\pm}
+
T_{-m\pm}
\right)
=
\frac{1}{4\pi}
\log\left\{
E_{\pm,+}
E_{\pm,-}
\right\},
\quad
\text{where}
\quad
E_{\ast,\pm}
=
1-\exp^{-\beta_{1}\vert x-\xo\vert \pm\ci\beta_{1}(y\ast\yo)}
.
\end{equation}
The logarithmic singularity is therefore contained in the series containing the difference of the $y$ coordinates, \ie the $y-\yo$ terms.
(Higher order Kummer transformation can be used to further accelerate convergence if required.)

The Green's function is now expressed as $G(\bx\vert\bxo)=$
\begin{equation}\nonumber
\hat{G}(\bx\vert\bxo)
+
\frac{1}{4\pi}
\sum_{\pm}
\log\left\{
E_{\pm,+}
E_{\pm,-}
\right\},
\;
\text{where}
\;
\hat{G}
=
\frac{1}{4\ci w}
\sum_{\pm}
\left(
S_{0\pm}
+
\sum_{m=1}^{\infty}
(S_{\pm m\pm}-T_{\pm m\pm})
\right).
\end{equation}

Discounting the case in which two disks touch, the source and field points will only meet as they traverse the same disk perimeter, \ie $\bx=\bx_{d}(\theta)$ and $\bxo=\bxo_{d}(\tho)$.
In this case the singularity can be transferred to a more convenient form by writing
\begin{equation}\nonumber
\log\left\{
E_{\pm,+}
E_{\pm,-}
\right\}
=
\log\left\{
\frac{E_{\pm,+}
E_{\pm,-}}{
\left(\frac{2\rad\pi}{w} \right)^2(1-\cos(\theta\pm \tho)
}
\right\}
+
\log\left\{
\left(
\frac{2\rad\pi}{w} 
\right)^2(1-\cos(\theta\pm \tho)
\right\}
.
\end{equation}
In the singular case $(-)$, the first term on the right-hand side of the above vanishes as the field and source points tend towards one another, \ie $\theta-\tho\to 0$, and the singularity is contained in the second term alone.
The double integral over the disk perimeter of the second term can be evaluated explicitly to be
\begin{equation}
\begin{array}{c}
\Big\langle\Big\langle
\log\left\{
\left(
\frac{2\rad\pi}{w} 
\right)^2(1-\cos(\theta\pm \tho)
\right\}
\Big\rangle\Big\rangle_{p,q}
=
\\[18pt]
\left\{
\begin{array}{c c}
\log(\sqrt{2}\rad\pi/w)
-
8\pi^{2}\log(2)
&
(p=q=0),
\\
\ds
-
\frac{1}{2\pi}
\int_{0}^{\pi}
\log(1-\cos\psi)
(1-\cos(p\psi))
\wrt\psi
-
8\pi^{2}\log(2)
&
(p=q\neq 0),
\\
0 & (p\neq q),
\end{array}
\right.
\end{array}
\end{equation}
where the integral appearing on the right-hand side requires numerical evaluation.





%

\subsubsection{Irregular frequencies}



\section{Wave tank}

To integrate the wave maker and the beach into the model, we separate the horizontal domain into three 
rectangular subdomains. These correspond to two exterior disk-free regions, denoted by $\Omega^{1} = 
\left\{x,y\,:\,0<x<x_1,\,0<y<w\right\}$ and $\Omega^{2} = \left\{x,y\,:\,x_2<x<l,\,0<y<w\right\}$, and an 
interior region containing all the disks, denoted by $\Omega^{\mathrm{d}} = \Omega \setminus 
\left(\Omega^{1} \cup \Omega^{2}\right)$. In the open-water region $\Omega^{i}$, $i=1,\,2$, the velocity 
potential, denoted by $\vp^{(i)}$, can be expressed as in \eqref{eq:PotOW}, \ie
\begin{equation}
\vp^{(i)}(x,y,z) \approx \sum_{n=0}^{N} \sum_{m=0}^{M} 
\Big(a^{(i)}_{n,m}\exp^{\ci\kx_{m}(k_{n})(x-x^{(i)}_+)} + 
b^{(i)}_{n,m}\exp^{-\ci\kx_{m}(k_{n}) (x-x^{(i)}_-)}\Big) \ym_{m}(y) \vm(z:k_{n}),
\end{equation}
where $a^{(i)}_{n,m}$ and $b^{(i)}_{n,m}$ are unknown amplitudes, and $x^{(1)}_+=0$, $x^{(1)}_-=x_1$, 
$x^{(2)}_+=x_2$ and $x^{(2)}_-=l$.

The solution in $\Omega^{\mathrm{d}}$ has been devised in \textsection \ref{sec2}. Let us denote by 
$\mathbf{S}$ the scattering matrix associated with $\Omega^{\mathrm{d}}$. It maps the incident waves 
amplitudes to the scattered waves amplitudes as follows
\begin{equation}
	\label{eq:disk_scat}
	\left(
	\begin{array}{c}
		\mathbf{b}^{(1)} \\
		\mathbf{a}^{(2)}
	\end{array}
	\right) = \left[
	\begin{array}{cc}
	\mathbf{S}^{11} & \mathbf{S}^{12} \\
	\mathbf{S}^{21} & \mathbf{S}^{22}
	\end{array}
	\right] \left(
	\begin{array}{c}
		\mathbf{a}^{(1)} \\
		\mathbf{b}^{(2)}
	\end{array}
	\right),
\end{equation}
where 
\[
	\mathbf{a}^{(i)} = \left(a^{(i)}_{0,0},\ldots,a^{(i)}_{N,0},a^{(i)}_{0,1},\ldots,a^{(i)}_{N,M-1},
	a^{(i)}_{0,M},\ldots,a^{(i)}_{N,M}\right)^{\mathrm{T}}
\]
and 
\[
	\mathbf{b}^{(i)} = \left(b^{(i)}_{0,0},\ldots,b^{(i)}_{N,0},b^{(i)}_{0,1},\ldots,b^{(i)}_{N,M-1},
	b^{(i)}_{0,M},\ldots,b^{(i)}_{N,M}\right)^{\mathrm{T}},
\]
$i=1,\,2$, have length $(N+1)(M+1)$.

Using matrix and vector notations, the velocity potential in $\Omega^{i}$, $i=1,\,2$, can be expressed as
\begin{equation}
  \label{eq:MatPot}
 \vp^{(i)}(x,y,z) \approx \mathbf{C}(y,z)\left(\mathbf{E}(x-x^{(i)}_+)\mathbf{a}^{(i)} + 
 \mathbf{E}(x^{(i)}_--x)\mathbf{b}^{(i)}\right),
\end{equation}
where
\[
  \mathbf{C}(y,z) = \left(C_{0,0},\ldots,C_{N,0},C_{0,1},\ldots,C_{N,M-1},C_{0,M},\ldots,C_{N,M}\right)
\]
has length $(N+1)(M+1)$, with 
\[
  C_{n,m}(y,z) = \ym_{m}(y) \vm(z:k_{n}).
\]
We have also introduced the diagonal matrix $\mathbf{E}(x)$ of size $(N+1)(M+1)$ defined by
\[
  \mathbf{E}(x) = \mathrm{diag}\left\{\exp^{\ci\kx_{0}(k_{0})x},\ldots,\exp^{\ci\kx_{0}(k_{N})x},
  \exp^{\ci\kx_{1}(k_{0})x},\ldots,\exp^{\ci\kx_{M-1}(k_{N})x},\exp^{\ci\kx_{M}(k_{0})x},\ldots,
  \exp^{\ci\kx_{M}(k_{N})x}\right\}.
\]


\subsection{Wave maker}

At $x=0$, a unidirectional wave maker of cross-section $\Lambda(z)$ sets the system in motion. The 
horizontal displacement of the paddle is described by $X_w(z,t)=\Lambda(z)X_0(t)$, where $X_0(t)$ is the 
time-dependent evolution of the paddle amplitude at $z = 0$. Assuming time-harmonic motion (\ie 
$X_0(t) = \real\{A_0\mathrm{exp}(\ci\omega t)\}$), the linearised 
boundary condition is given by
\begin{equation}
  \label{eq:WM}
  \ci \omega A_0 \Lambda(z) = \partial_x \vp^{(1)} \quad (x=0).
\end{equation}

Using \eqref{eq:MatPot}, we multiply \eqref{eq:WM} by $\mathbf{C}^{\mathrm{T}}(y,z)$ and integrate over the 
vertical and transversal domains. This yields
\begin{equation}
  \label{eq:WM_proj}
  \mathbf{F} = \mathbf{B}\left(\mathbf{E}'(0)\mathbf{a}^{(1)} + \mathbf{E}'(x_1)\mathbf{b}^{(1)}\right),
\end{equation}
where
\[
  \mathbf{F} = \ci \omega A_0 \int_{-H}^0 \int_0^w \Lambda(z) \mathbf{C}^{\mathrm{T}}(y,z) \wrt y \wrt z
\]
has length $(N+1)(M+1)$ and entries given by 
\[
  \big[\mathbf{F}\big]_{n,m} = \ci \omega A_0 w \delta_{0,m} \int_{-H}^0 \Lambda(z) \vm(z:k_{n}) \wrt z.
\]
We have also introduced the $(N+1)(M+1)$-size matrix
\[
  \mathbf{B} = \int_{-H}^0 \int_0^w \mathbf{C}^{\mathrm{T}}(y,z) \mathbf{C}(y,z) \wrt y \wrt z,
\]
which is diagonal with entries $w/2$, from the orthogonality properties of $C_{n,m}(y,z)$.

Re-arranging \eqref{eq:WM_proj} yields 
\begin{equation}
  \label{eq:WM_scat}
  \mathbf{a}^{(1)} = \mathbf{S}_w\mathbf{b}^{(1)} + \mathbf{F}_w,
\end{equation}
where the scattering matrix $\mathbf{S}_w$ and the forcing vector $\mathbf{F}_w$ are obtained 
straightforwardly.


\subsection{Beach}

At $x=l$, the beach partly attenuates waves, depending on their frequency. It is difficult to model a beach 
properly within the framework of the linearised theory of water waves, as the attenuation process is 
non-linear. Instead, its effect on the system is considered assuming the beach is sufficiently far from 
other scatterers, so that it only influences propagating waves. This can be achieved by modelling the beach 
as a piston-like wave absorber. A kinematic condition is then prescribed as
\begin{equation}
  \label{eq:beach}
  \ci \omega A_b = \partial_x \vp^{(2)} \quad (x=l),
\end{equation}
where $A_b$ is the horizontal displacement of the paddle. The control law of the wave absorber is defined 
such that a single travelling vertical mode is generated upon reflection at $x=l$, \ie
\begin{equation}
  \label{eq:beach_scat}
  \mathbf{b}^{(2)} = \mathbf{R}_b\mathbf{E}(l-x_2)\mathbf{a}^{(2)} = \mathbf{S}_b\mathbf{a}^{(2)},
\end{equation}
where $\mathbf{R}_b$ is the reflection $(N+1)(M+1)$-size diagonal matrix with entries $\mathcal{R}_b$ if 
$n=0$ and $0$ otherwise. The reflection coefficient $\mathcal{R}_b$ is governed by the beach spectrum 
(Montiel \etal \cite{Montiel_etal12})
\begin{equation}
  \mathcal{R}_b = 0.9\left(\exp^{-5\omega} -1\right) + 1.
\end{equation}

The flow in the wave tank is then governed by the matrix equations \eqref{eq:disk_scat}, \eqref{eq:WM_scat} 
and \eqref{eq:beach_scat}. The forcing is input by vector $\mathbf{F}_w$ in \eqref{eq:WM_scat}. Calculation 
of the unknown amplitude vectors $\mathbf{a}^{(i)}$ and $\mathbf{b}^{(i)}$, $i=1,\,2$, follows 
straightforwardly.


%%%%%%%%%%%%%%%%%%%%%%%%%%%%%%%%%%%%%%%%%%%%%%%%%%%

\bibliographystyle{plain}      % References
% \bibliography{/Users/a1612881/Dropbox/Documents/bibli/LukesBibli}
\bibliography{biblio}

%\begin{thebibliography}{10}
%
%\end{thebibliography}


\end{document}
